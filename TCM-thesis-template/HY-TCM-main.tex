%% Follow comments to support use.

%%%%%%%%%%%%%%%%%%%%%%%%%%%%%%%%%%%%%%%%%%%%%%%%%%%%%%%%%
%% STEP 1: Choose options for MSc layout and your bibliographic style
%%%%%%%%%%%%%%%%%%%%%%%%%%%%%%%%%%%%%%%%%%%%%%%%%%%%%%%%%

%%  Language: 
%%      finnish, swedish, or english
%%  Pagination (use twoside by default)  
%%      oneside or twoside,
%%  Study programme
%%      tcm  = Master's thesis in Master's Programme in Theoretical and computational Methods;

\documentclass[english,twoside,tcm]{HYthesisTCM} 


% If wanted, open new chapters only at right page.
% By default, "openany".
%\PassOptionsToClass{openright,twoside,a4paper}{report}
\PassOptionsToClass{openany,twoside,a4paper}{report}

\usepackage{csquotes}
%%%%%%%%%%%%%%%%%%%%%%%%%%%%%%%%%%%%%%%%%%%%%%%%%%%%%%%%%
%% REFERENCES
%% Some notes on bibliography usage and options:
%% natbib -> you can use, e.g., \citep{} or \parencite{} for (Einstein, 1905); with APA \cite -> Einstein, 1905 without ()
%% maxcitenames=2 -> only 2 author names in text citations, if more -> et al. is used
%% maxbibnames=99 as no great need to suppress the biliography list in a thesis
%% for more information see biblatex package documentation, e.g., from https://ctan.org/pkg/biblatex 

%% Reference style: select one 
%% for APA = Harvard style = authoryear -> (Einstein, 1905) use:
% \usepackage[style=authoryear,bibstyle=authoryear,backend=biber,natbib=true,maxnames=99,maxcitenames=2,uniquelist=minyear,giveninits=true,uniquename=mininit]{biblatex}
%% for numeric = Vancouver style -> [1] use:
\usepackage[style=numeric,bibstyle=numeric,backend=biber,natbib=true,maxbibnames=99,giveninits=true,uniquename=init]{biblatex}
%% for alpahbetic -> [Ein05] use:
% \usepackage[style=alphabetic,bibstyle=alphabetic,backend=biber,natbib=true,maxbibnames=99,giveninits=true,uniquename=init]{biblatex}
%

\addbibresource{bibliography.bib}
% in case you want the final delimiter between authors & -> (Einstein & Zweistein, 1905) 
% \renewcommand{\finalnamedelim}{ \& }
% List the authors in the Bibilipgraphy as Lastname F, Familyname G,
\DeclareNameAlias{sortname}{family-given}
% remove the punctuation between author names in Bibliography 
%\renewcommand{\revsdnamepunct}{ }


%% Block of definitions for fonts and packages for picture management.
%% In some systems, the figure packages may not be happy together.
%% Choose the ones you need.

%\usepackage[utf8]{inputenc} % For UTF8 support, in some systems. Use UTF8 when saving your file.

\usepackage{lmodern}         % Font package, again in some systems.
\usepackage{textcomp}        % Package for special symbols
\usepackage[pdftex]{color, graphicx} % For pdf output and jpg/png graphics
\usepackage{epsfig}
\usepackage{subfigure}
\usepackage[pdftex, plainpages=false]{hyperref} % For hyperlinks and pdf metadata
\usepackage{fancyhdr}        % For nicer page headers
\usepackage{tikz}            % For making vector graphics (hard to learn but powerful)
%\usepackage{wrapfig}        % For nice text-wrapping figures (use at own discretion)
\usepackage{amsmath, amssymb} % For better math

\singlespacing               %line spacing options; normally use single

\fussy
%\sloppy                      % sloppy and fussy commands can be used to avoid overlong text lines
% if you want to see which lines are too long or have too little stuff, comment out the following lines
% \overfullrule=1mm
% to see more info in the detailed log about under/overfull boxes...
% \showboxbreadth=50 
% \showboxdepth=50


%%%%%%%%%%%%%%%%%%%%%%%%%%%%%%%%%%%%%%%%%%%%%%%%%%%%%%%%%
%% STEP 2:
%%%%%%%%%%%%%%%%%%%%%%%%%%%%%%%%%%%%%%%%%%%%%%%%%%%%%%%%%
%% Set up personal information for the title page and the abstract form.
%% Replace parameters with your information.
\title{Title}

\author{Firstname Lastname}
\date{\today}

% Set supervisors, use the titles according to the thesis language
% in English Prof. or Dr., or in Finnish toht. or tri or FT, TkT, Ph.D. or in Swedish... 
\supervisors{Prof.~D.U.~Mind, Dr.~O.~Why}

% \keywords{algorithms, data structures}
% \additionalinformation{\translate{\track}}
% \additionalinformation{Any additional information.}


%% If you want to quote someone special. You can comment this line out and there will be nothing on the document.
\quoting{Bachelor's degrees make pretty good placemats if you get them laminated.}{Jeph Jacques}


%% OPTIONAL STEP: Set up properties and metadata for the pdf file that pdfLaTeX makes.
%% Your name, work title, and keywords are recommended.
\hypersetup{
    unicode=true,           % to show non-Latin characters in Acrobat’s bookmarks
    pdftoolbar=true,        % show Acrobat’s toolbar?
    pdfmenubar=true,        % show Acrobat’s menu?
    pdffitwindow=false,     % window fit to page when opened
    pdfstartview={FitH},    % fits the width of the page to the window
    pdftitle={},            % title
    pdfauthor={},           % author
    pdfsubject={},          % subject of the document
    pdfcreator={},          % creator of the document
    pdfproducer={pdfLaTeX}, % producer of the document
    pdfkeywords={something} {something else}, % list of keywords for
    pdfnewwindow=true,      % links in new window
    colorlinks=true,        % false: boxed links; true: colored links
    linkcolor=black,        % color of internal links
    citecolor=black,        % color of links to bibliography
    filecolor=magenta,      % color of file links
    urlcolor=cyan           % color of external links
}

%%-----------------------------------------------------------------------------------

\begin{document}
\frontmatter
% Generate title page.
\maketitle

%%%%%%%%%%%%%%%%%%%%%%%%%%%%%%%%%%%%%%%%%%%%%%%%%%%%%%%%%
%% STEP 3:
%%%%%%%%%%%%%%%%%%%%%%%%%%%%%%%%%%%%%%%%%%%%%%%%%%%%%%%%%
%% Write your abstract in the separate file, to be positioned here.
%% You can make several abstract pages (if you want it in different languages),
%% in which case you should also define the language of the abstract,
%% as below.

\keywords{algorithms, data structures}
\additionalinformation{Any additional information.}
\begin{abstract}

Write your abstract here.
\end{abstract}


%% Uncomment the following lines if you want to include a second abstract in another language.

% \keywords{algoritmit, data rakenteet}
% \additionalinformation{Muita lisätietoja.}
% \begin{otherlanguage}{finnish}
% \begin{abstract}

% Kirjoita tiivistelmä tähän.

% \end{abstract}
% \end{otherlanguage}
\myquote

%% Uncomment the following line if you want to include a preface in your thesis
\preface{I would like to thank$\dots$}

%% Uncomment the following lines if you want to include a list of symbols in your thesis
\renewcommand\nomgroup[1]{%
  \item[\bfseries
  \ifstrequal{#1}{B}{Coplanar waveguide resonator}{%
  \ifstrequal{#1}{C}{Josephson junction}{%
  \ifstrequal{#1}{A}{Constants }{}}}%
]}
\renewcommand{\nompreamble}{Some explanation about the list of symbols.}
\nomenclature[B, 01]{CPW}{Coplanar waveguide \nomunit{}}
\nomenclature[B, 02]{$2l$}{Length of the center conductor \nomunit{m}}
\nomenclature[C, 01]{$x_\text{J}$}{Location of the Josephson junction \nomunit{m}}
\nomenclature[C, 02]{$C_\text{J}$}{Capacitance of the Josephson junction \nomunit{F}}
\nomenclature[A, 01]{$h$}{Planck constant \nomunit{$6.62607014 \cdot 10^{-34}$ Js}}
\nomenclature[A, 02]{$e$}{Elementary charge \nomunit{$1.602176634 \cdot 10^{-19}$ C}}
\mynomenclature

% Place ToC
%\newpage
\mytableofcontents


\mainmatter

%%%%%%%%%%%%%%%%%%%%%%%%%%%%%%%%%%%%%%%%%%%%%%%%%%%%%%%%%
%% STEP 4: Write the thesis.
%%%%%%%%%%%%%%%%%%%%%%%%%%%%%%%%%%%%%%%%%%%%%%%%%%%%%%%%%
%% Your actual text starts here. You shouldn't mess with the code above the line except
%% to change the parameters. Removing the abstract and ToC commands will mess up stuff.
%%
%% Command \include{file} includes the file of name file.tex.
%% A new page will be created at every \include command, 
%% which makes it appropriate to use it for large entities such as book chapters. Cannot be nested.
%% It is useful for a big project, as changing one of the include targets 
%% won't force the regeneration of the outputs of all the rest.
%% Alternatively, \input is a more lower level macro 
%% which simply inputs the content of the given file like it was copy&pasted there manually.

\include{Ch.10_Introduction}
\include{Ch.20_Methods}
\include{Ch.30_Results}
\include{Ch.40_Discussion}
\include{Ch.50_Conclusions}


%%%%%%%%%%%%%%%%%%%%%%%%%%%%%%%%%%%%%%%%%%%%%%%%%%%%%%%%%
%\cleardoublepage                          %fixes the position of bibliography in bookmarks
%\phantomsection
\addcontentsline{toc}{chapter}{\bibname}  % This lines adds the bibliography to the ToC
\markboth{\bibname}{}%to fix header in case of multi-page bibliography
\printbibliography

%%%%%%%%%%%%%%%%%%%%%%%%%%%%%%%%%%%%%%%%%%%%%%%%%%%%%%%%%
\backmatter
\begin{appendices}

%% A sample Appendix
\include{Ch.90_Appendix_1}
%% another appendix

\appendix{Instructions for LaTex}

\section{General Setup}

In the HY-Physics-main.tex file you will find instructions of how to use the template as comments. The instructions are divided into STEPS 1--4. Below you can find related instructions.
\vspace{0.5cm}

\textbf{STEP 0 -- Access the thesis template}

\begin{itemize}
\item Import the thesis template into a new Overleaf project. The easiest way to do it is to:
\begin{itemize}
    \item Click on the name of the template on the top of the Overleaf page and choose ''Make a copy''. This will create a copy of the template that you can edit and is only visible for you.
    \item Make sure you have connected your University of Helsinki email to your Overleaf account to access useful tools.
\end{itemize}
\end{itemize}

\textbf{STEP 1 -- Choose options for the document and your bibliographic style}
\begin{itemize}
    \item In the \texttt{documentclass} command you can choose:
    \begin{itemize}
        \item The language of your thesis.
        \item Your Master's programme.
        \item Your study track (not TCM)
        \item The number of supervisors you have.
    \end{itemize}
    \item Choose your bibliography style. The default style is numbered [1], but it can be easily changed to Author-Year (Einstein, 1905) or alphabetical [Ein05], as the examples of these are in comments. Remember to discuss the style to use with your supervisor. See App.~\ref{app:bibliography} for details.
\end{itemize}


{\textbf{STEP 2 -- Set up your personal information}}

\begin{enumerate}
\item Specify the title of your thesis with \texttt{\textbackslash title\{\}}.
\item Specify your name to the author field with \texttt{\textbackslash author\{\}}.
\item Specify the names of your supervisors of the thesis with \texttt{\textbackslash supervisors\{\}}.
\end{enumerate}

{\textbf{STEP 3 -- Write your abstract}}

\begin{itemize}
\item Write your abstract and choose keywords with \texttt{\textbackslash keywords\{\}}. You should probably do this only after you have written most of your thesis.
\item You can have the abstract in multiple languages with the \texttt{otherlanguages} environment. The example below shows how to provide an English abstract: 

\begin{verbatim}
\begin{otherlanguage}{english} 
\begin{abstract}
Your abstract text goes here. 
\end{abstract} 
\end{otherlanguage}
\end{verbatim}

\end{itemize}

{\textbf{STEP 4 -- Writing your thesis}}

\begin{enumerate}
\item It is usually a good idea to write each chapter in its own \texttt{.tex} file and use either \texttt{\textbackslash include\{\}} or \texttt{\textbackslash input\{\}} to insert them into the main document.
\item There are some additional instructions below about writing in \LaTeX and including figures and tables.
\item Remove, or comment out, this appendix from your thesis.
\end{enumerate}



\section{Bibliography in Latex}\label{app:bibliography}

The bibliography is defined in a separate \texttt{.bib} file. For this template, it is named\\
\texttt{bibliography.bib} and includes the content show in Figure~\ref{bibexamples}.

Chapter Bibliography lists all the works that you refer to in your text. You refer to the works in the bibliography using an appropriate \emph{citation key}.
%
%This thesis template contains an example of a bibliography.


References are done using \texttt{\textbackslash citep\{einstein\}}, which generates in text a citation formatted according to the selected style \citep{einstein}, or \texttt{\textbackslash citep\{latexcompanion,knuth99\}}, which generates \citep{latexcompanion,knuth99}. 
As examples of a different kinds of citations (see how these look in the Latex source), we can write \citep{einstein} to refer to the work written by \citeauthor{einstein} in \citeyear{einstein}, because the work by \citet{einstein} appears in the bibliography included in this template.

Note that there are different possible styles for the bibliography and citation keys.
%
Consult your supervisors on the chosen style---once you arrive at a preferred style, use it consistently throughout the thesis.

\begin{figure}[ht]
    \centering
    \begin{scriptsize}
\begin{verbatim}
@article{einstein,
    author =       "Albert Einstein",
    title =        "{Zur Elektrodynamik bewegter K{\"o}rper}. ({German})
        [{On} the electrodynamics of moving bodies]",
    journal =      "Annalen der Physik",
    volume =       "322",
    number =       "10",
    pages =        "891--921",
    year =         "1905",
    DOI =          "http://dx.doi.org/10.1002/andp.19053221004"
}
 
@book{latexcompanion,
    author    = "Michel Goossens and Frank Mittelbach and Alexander Samarin",
    title     = "The \LaTeX\ Companion",
    year      = "1993",
    publisher = "Addison-Wesley",
    address   = "Reading, Massachusetts"
}

@book{knuth99,
    author    = "Donald E. Knuth",
    title     = "Digital Typography",
    year      = "1999",
    publisher = "The Center for the Study of Language and Information",
    series    = "CLSI Lecture Notes (78)"
}\end{verbatim}
\end{scriptsize}
    \caption{Examples of bibliographic reference in .bib file.}
    \label{bibexamples}
\end{figure}

%In the last reference url field the code \verb+%7E+ will translate into \verb+~+ once clicked in the final pdf.

\section{Some instructions about writing in Latex}

The following gives some superficial instructions for using this template for a Master's thesis. For guidelines on thesis writing you can consult various sources, such as university courses on scientific writing or your supervisors.

For more detailed instructions, just google, e.g., ''Overleaf table positioning'', and your chances of finding good info are pretty good.  


\section{Figures}
Besides text, here are simple examples how you can add figures and tables in your thesis.
Remember always to refer to each figure in the main text and provide them with a descriptive caption.

Figure~\ref{fig:logo} is an example of a figure in the document (see the source about how to add them). 
%Using figures is particularly useful to display plots of experimental results.

\begin{figure}[ht] 
\begin{center}
\includegraphics[width=0.3\textwidth]{template/figures/HY-logo-ml.png}
\caption{University of Helsinki flame-logo for Faculty of Science.\label{fig:logo}}
\end{center}
\end{figure}

\section{Tables}

Table~\ref{table:results} gives an example of a table.
Remember always to cite the table in the main text, table captions go on top of the table. 

\begin{table}[h] % h positions the table here, t! would force on top of the page, or example.
\begin{center}
\caption{Experimental results.\label{table:results}} % caption is here to make it on top
\begin{tabular}{l||l c r} 
Experiment & 1 & 2 & 3 \\ 
\hline \hline 
$A$ & 2.5 & 4.7 & -11 \\
$B$ & 8.0 & -3.7 & 12.6 \\
$A+B$ & 10.5 & 1.0 & 1.6 \\
\hline
%
\end{tabular}
\end{center}
\end{table}





\end{appendices}
%%%%%%%%%%%%%%%%%%%%%%%%%%%%%%%%%%%%%%%%%%%%%%%%%%%%%%%%%

\end{document}
