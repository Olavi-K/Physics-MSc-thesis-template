
\appendix{Instructions for LaTex}

\section{General Setup}

In the HY-Physics-main.tex file you will find instructions of how to use the template as comments. The instructions are divided into STEPS 1--4. Below you can find related instructions.
\vspace{0.5cm}

\textbf{STEP 0 -- Access the thesis template}

\begin{itemize}
\item Import the thesis template into a new Overleaf project. The easiest way to do it is to:
\begin{itemize}
    \item Click on the name of the template on the top of the Overleaf page and choose ''Make a copy''. This will create a copy of the template that you can edit and is only visible for you.
    \item Make sure you have connected your University of Helsinki email to your Overleaf account to access useful tools.
\end{itemize}
\end{itemize}

\textbf{STEP 1 -- Choose options for the document and your bibliographic style}
\begin{itemize}
    \item In the \texttt{documentclass} command you can choose:
    \begin{itemize}
        \item The language of your thesis.
        \item Your Master's programme.
        \item Your study track (not TCM)
        \item The number of supervisors you have.
    \end{itemize}
    \item Choose your bibliography style. The default style is numbered [1], but it can be easily changed to Author-Year (Einstein, 1905) or alphabetical [Ein05], as the examples of these are in comments. Remember to discuss the style to use with your supervisor. See App.~\ref{app:bibliography} for details.
\end{itemize}


{\textbf{STEP 2 -- Set up your personal information}}

\begin{enumerate}
\item Specify the title of your thesis with \texttt{\textbackslash title\{\}}.
\item Specify your name to the author field with \texttt{\textbackslash author\{\}}.
\item Specify the names of your supervisors of the thesis with \texttt{\textbackslash supervisors\{\}}.
\end{enumerate}

{\textbf{STEP 3 -- Write your abstract}}

\begin{itemize}
\item Write your abstract and choose keywords with \texttt{\textbackslash keywords\{\}}. You should probably do this only after you have written most of your thesis.
\item You can have the abstract in multiple languages with the \texttt{otherlanguages} environment. The example below shows how to provide an English abstract: 

\begin{verbatim}
\begin{otherlanguage}{english} 
\begin{abstract}
Your abstract text goes here. 
\end{abstract} 
\end{otherlanguage}
\end{verbatim}

\end{itemize}

{\textbf{STEP 4 -- Writing your thesis}}

\begin{enumerate}
\item It is usually a good idea to write each chapter in its own \texttt{.tex} file and use either \texttt{\textbackslash include\{\}} or \texttt{\textbackslash input\{\}} to insert them into the main document.
\item There are some additional instructions below about writing in \LaTeX and including figures and tables.
\item Remove, or comment out, this appendix from your thesis.
\end{enumerate}



\section{Bibliography in Latex}\label{app:bibliography}

The bibliography is defined in a separate \texttt{.bib} file. For this template, it is named\\
\texttt{bibliography.bib} and includes the content show in Figure~\ref{bibexamples}.

Chapter Bibliography lists all the works that you refer to in your text. You refer to the works in the bibliography using an appropriate \emph{citation key}.
%
%This thesis template contains an example of a bibliography.


References are done using \texttt{\textbackslash citep\{einstein\}}, which generates in text a citation formatted according to the selected style \citep{einstein}, or \texttt{\textbackslash citep\{latexcompanion,knuth99\}}, which generates \citep{latexcompanion,knuth99}. 
As examples of a different kinds of citations (see how these look in the Latex source), we can write \citep{einstein} to refer to the work written by \citeauthor{einstein} in \citeyear{einstein}, because the work by \citet{einstein} appears in the bibliography included in this template.

Note that there are different possible styles for the bibliography and citation keys.
%
Consult your supervisors on the chosen style---once you arrive at a preferred style, use it consistently throughout the thesis.

\begin{figure}[ht]
    \centering
    \begin{scriptsize}
\begin{verbatim}
@article{einstein,
    author =       "Albert Einstein",
    title =        "{Zur Elektrodynamik bewegter K{\"o}rper}. ({German})
        [{On} the electrodynamics of moving bodies]",
    journal =      "Annalen der Physik",
    volume =       "322",
    number =       "10",
    pages =        "891--921",
    year =         "1905",
    DOI =          "http://dx.doi.org/10.1002/andp.19053221004"
}
 
@book{latexcompanion,
    author    = "Michel Goossens and Frank Mittelbach and Alexander Samarin",
    title     = "The \LaTeX\ Companion",
    year      = "1993",
    publisher = "Addison-Wesley",
    address   = "Reading, Massachusetts"
}

@book{knuth99,
    author    = "Donald E. Knuth",
    title     = "Digital Typography",
    year      = "1999",
    publisher = "The Center for the Study of Language and Information",
    series    = "CLSI Lecture Notes (78)"
}\end{verbatim}
\end{scriptsize}
    \caption{Examples of bibliographic reference in .bib file.}
    \label{bibexamples}
\end{figure}

%In the last reference url field the code \verb+%7E+ will translate into \verb+~+ once clicked in the final pdf.

\section{Some instructions about writing in Latex}

The following gives some superficial instructions for using this template for a Master's thesis. For guidelines on thesis writing you can consult various sources, such as university courses on scientific writing or your supervisors.

For more detailed instructions, just google, e.g., ''Overleaf table positioning'', and your chances of finding good info are pretty good.  


\section{Figures}
Besides text, here are simple examples how you can add figures and tables in your thesis.
Remember always to refer to each figure in the main text and provide them with a descriptive caption.

Figure~\ref{fig:logo} is an example of a figure in the document (see the source about how to add them). 
%Using figures is particularly useful to display plots of experimental results.

\begin{figure}[ht] 
\begin{center}
\includegraphics[width=0.3\textwidth]{template/figures/HY-logo-ml.png}
\caption{University of Helsinki flame-logo for Faculty of Science.\label{fig:logo}}
\end{center}
\end{figure}

\section{Tables}

Table~\ref{table:results} gives an example of a table.
Remember always to cite the table in the main text, table captions go on top of the table. 

\begin{table}[h] % h positions the table here, t! would force on top of the page, or example.
\begin{center}
\caption{Experimental results.\label{table:results}} % caption is here to make it on top
\begin{tabular}{l||l c r} 
Experiment & 1 & 2 & 3 \\ 
\hline \hline 
$A$ & 2.5 & 4.7 & -11 \\
$B$ & 8.0 & -3.7 & 12.6 \\
$A+B$ & 10.5 & 1.0 & 1.6 \\
\hline
%
\end{tabular}
\end{center}
\end{table}


